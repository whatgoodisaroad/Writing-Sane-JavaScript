\documentclass[11pt,letter]{book}

\addtolength{\oddsidemargin}{-.875in}
\addtolength{\evensidemargin}{-.875in}
\addtolength{\textwidth}{1.75in}
\addtolength{\topmargin}{-.875in}
\addtolength{\textheight}{1.75in}

\begin{document}
    \title{Writing Sane JavaScript}
    \author{Wyatt Allen}
    \maketitle
    
    \chapter{Introduction}
    \begin{quote}
        \emph{
            ...an interesting mix ... capable of such beautiful dreams and such horrible nightmares.
        } \\
        -- Carl Sagan, \emph{Contact}
    \end{quote}
    
    JavaScript is a truly remarkable language. It is ubiquitous: every computer with a browser 
    installed comes with a full runtime. It's uncommonly powerful: its dynamic nature is more 
    flexible than nearly any enterprise language, yet is used to do more and more amazing things.
    It's unstoppable: since it's halfhearted genesis in 1995 it's never slowed it's growing 
    popularity.
    
    Moreover, JavaScript is both beautiful and deeply flawed.
    
    TODO: Title, History, Significance, Beauty, State, How to read
    
    \part{The Language}
    \chapter{The Type System}
    \section{Overview}
    The key to mastering JavaScript is a complete understanding of the type system. In this chapter
    we will have a first look at each of the fundamental datatypes in JavaScript. We examine the 
    type system before presenting any code examples, so this chapter may seem a little dry. Let the 
    reader be ensured, however, that this approach is to provide a rich context for when we finally
    do get to code.
    
    JavaScript has a unique type system. In many ways it is Spartan. In many other ways 
    it is rich and expressive. In many, many other ways it is inconsistent and treacherous.
    Throughout this book, you will be introduced to the type system from each of these three 
    angles.
    
    If you, the reader, gain a complete understanding of these types, it will either result 
    in true enjoyment of a unique and powerful system, or well-informed hatred of a tangled 
    mess. Read on to find out which it will be.
    
    \section{Datatypes at a Fundamental Level}
    At a fundamental level JavaScript has four basic datatypes. There are three primitive 
    types which should be familiar to you as a programmer: \texttt{boolean}, \texttt{number} 
    and \texttt{string}. The fourth datatype is \texttt{object}, which may be thought of as a 
    composite type. Although the three primitives will behave largely in a similar 
    fashion to what you're used to in other languages, some of \texttt{object}'s nuances 
    may be surprising.
    
    Although I said there were only three, JavaScript actually defines two additional types: which
    are technically primitive also, \texttt{null} and \texttt{undefined}. \texttt{null} will 
    probably look familiar to you from most any object-oriented language. \texttt{undefined} might 
    be new to you. We'll explore these types in greater detail (and see why they're very different 
    from the other primitives) once we've dug deeper into the type system. For now, just put them on 
    your radar.
    
    It's important to note that the line between primitive and object is not so clearly defined
    in JavaScript as it is in other languages. You'll find that nearly any value in JavaScript can
    \emph{behave} like an object while remaining primitive. Some people interpret this to mean that
    \emph{every value} in JavaScript is an object in disguise, but this is not strictly the case.
    We'll look at objects very closely soon enough.
    
    \subsection{The \texttt{boolean} Datatype}
    If we restrict ourselves to the fundamentals of the type system, the \texttt{boolean} 
    datatype is as simple as can be. The ECMAScript standard explains it in a single sentence:
    
    \begin{quote}
        The Boolean type represents a logical entity having two values, called \textbf{true}
        and \textbf{false}.
    \end{quote}
    
    In practice, \texttt{true} and \texttt{false} are keywords which  statically refer to the two 
    possible boolean values (just as you find in most programming languages). These are considered 
    ``Boolean Literals.''
    
    \subsection{The \texttt{number} Datatype}
    In JavaScript, there is no distinction between integral and floating-point numbers. Instead, 
    we're given a single, unified type: \texttt{number}.
    
    A number is stored as a signed 64-bit float according to the IEEE 754 standard. This means, 
    under the covers, the numbers you deal with in JavaScript have the same encoding constraints as 
    double-precision floats in Java or C.
    
    It is worth noting that the simplicity you get by having a unified type for numbers comes at 
    a cost of space-efficiency, performance and type-safety. On the other hand this will make 
    some of your code cleaner because you will no-longer need no coerce integers to floats or 
    worry about loss of precision in mixed-mode division. In general, we will find that 
    JavaScript tends to sacrifice efficiency/performance/safety in small ways in order to gain 
    in simplicity.
    
    A JavaScript number can be as large as $\pm 1.7976931348623157 \times 10^{308}$ or as small as
    $\pm 5 \times 10^{-324}$. The range of integral values which can be stored in a JavaScript 
    \texttt{number} is $\pm 2^{53}$.
    
    Numbers can be declared with number literals. Number literals can be base-10 or base-16 (using 
    the traditional \texttt{0x} prefix). In some implementations base-8 number literals are allowed 
    when the number is prefixed with a leading zero. Number literals can also be written with 
    exponential notation by suffixing a lower or upper-case letter ``e'' followed by a positive or 
    negative radix. 
    
    Negative number literals can be made in the usual way, by prefixing the literal with a minus 
    sign. However, this isn't actually part of the literal itself, but is an application of the 
    unary negation operator to the literal.
    
    Here are some examples of simple number values:
    
    \begin{verbatim}
    7
    
    42
    
    -512
    
    7.62
    
    6.022141e23
    
    6.6260695729E-34
    
    0xf0
    
    0xff63ba
    
    \end{verbatim}
    
    \subsection{The \texttt{string} Datatype}
    JavaScript provides \texttt{string} as a primitive type for dealing with text. All strings in 
    JavaScript are encoded using UTF-16.
    
    You just learned that JavaScript unifies integers and floats under numbers at a cost of 
    performance and for the sake of simplicity. With strings, JavaScript makes a similar 
    simplification: there is no \texttt{char} type. That is to say: in JavaScript any single 
    character is of type \texttt{string} with a string-length of 1. For dealing with  
    text, the \texttt{string} datatype is the only datatype you need.
    
    String literals must be delimited by matching single or double quotes. In other words, strings 
    may be delimited by either a matching start and end single quote or a matching start and end 
    double quote but not mixed. The quote type which is not used to delimit may be used within the 
    string literal without escaping.
    
    JavaScript \texttt{string}s support many of the escape sequences you're used to.
    
    \vspace{10pt}
    \begin{tabular}{l|l}
        Sequence & Result \\
        \hline
        \texttt{\textbackslash{'}}  & A single quote. (\textbackslash{u0027}) \\
        \texttt{\textbackslash{"}}  & A double quote. (\textbackslash{u0022}) \\
        \texttt{\textbackslash\textbackslash} 
                                    & A backslash. (\textbackslash{u005c}) \\
        \texttt{\textbackslash{u\#\#\#\#}}
                                    & The unicode character represented by the hexadecimal
                                        digits \#\#\#\#. \\
        \texttt{\textbackslash{t}}  & A horizontal tab. (\textbackslash{u0009}) \\
        \texttt{\textbackslash{n}}  & A newline. (\textbackslash{u000a}) \\
    \end{tabular}
    \vspace{10pt}
    
    Several other escape sequences are available, but are not useful in most JavaScript scenarios. 
    A full list of available escape sequences can be found in the appendix.
    
    Here are some examples of simple string values:
    
    \begin{verbatim}
    "Hello, world."
    
    'Hello, world with single quotes.'
    
    "This apostrophe needn't be escaped."
    
    'This apostrophe\'s escape is necessary.'
    
    'Double "quotes in single" quotes.'
    
    "Text\n spanning\n multiple\n lines"
    
    \end{verbatim}
    
    \subsection{The \texttt{object} Datatype}
    
    As stated above, the \texttt{object} datatype may be thought of as a composite of other 
    datatypes. In particular, an object is a collection of \emph{named data properties}, similar
    to a dictionary you may be familiar with from other languages.
    
    It is a remarkable feature of JavaScript that the \texttt{object} datatype also has a literal
    notation. The \texttt{object} is delimited by a pair of curly braces (i.e. ``\texttt{\{}'' and 
    ``\texttt{\}}''). Within these braces, any number of named data properties are specified by a 
    name, followed by a colon (i.e. \texttt{:}) followed by the property's data value. These named 
    data properties must be separated by commas.
    
    As an example, the following is an object literal with two properties: a property named 
    \texttt{foo} with the value of the \texttt{number} \texttt{42} and another property named 
    \texttt{bar} with the value of the \texttt{string} \texttt{"hello, world"}.
    
    \begin{verbatim}
    { foo:42, bar:"hello, world" }
    
    \end{verbatim}
    
    The order of the properties within the object literal is insignificant because the properties
    will be accessed by their names. For example, if this object were stored in the variable 
    \texttt{x}, the \texttt{foo} property could be accessed via the familiar \emph{dot-notation}.
    
    \begin{verbatim}
    x.foo
    
    \end{verbatim}
    
    There is an additional \emph{bracket-notation} for accessing properties by a string for their 
    name. The following example accesses the same property as the example above, but with the 
    alternate notation.
    
    \begin{verbatim}
    x["foo"]
    
    \end{verbatim}
    
    At this point objects may feel significantly similar to a conventional dictionary collection;
    string-keys map to values. This bracket-natation makes two big things possible. First: it allows 
    you to access properties of an object dynamically, via a string variable. Later on, we'll loop
    over the properties of an object by enumerating it's property names into a key variable, and 
    then access the property values themselves by using that key variable with the bracket-natation.
    
    Secondly, the bracket-natation makes possible the accessing of properties which have names with 
    symbols that are not valid in normal property names. For example, it is a single property named 
    \texttt{prop.erty} cannot be accessed using the dot notation because the dot in its name will be
    misinterpreted as accessing the \texttt{erty} subproperty of the \texttt{prop} property. Instead
    we can access this value in the following manner.
    
    \begin{verbatim}
    x["prop.erty"]
    
    \end{verbatim}
    
    As you can see, because the dot is safely enclosed within the string, JavaScript can see that it 
    is the name of a single property.
    
    The \texttt{object literal} syntax also allows illegal symbols in property names if you write
    the property name as a string. For example:
    
    \begin{verbatim}
    { "name with spaces":"hello, world", "(){}[]":"another value" }
    
    \end{verbatim}
    
    You may notice that all of these bracket-natation examples use strings as keys, and may wonder
    whether other datatypes may be used for property names. The answer is yes, sortof. We'll explore 
    the use of non-strings as property indexers when we take a much deeper look at the object 
    datatype.
    
    One must make an important compatability note that the final named property should not be 
    followed by a terminal comma. In seems harmless, and most browsers it will be tolerated, but 
    some versions of Internet Explorer will tank your script when they encounter a final comma.
    
    \subsection{Built-in Objects}
    It may seem odd that we discuss the four fundamental datatypes while not looking at arrays.
    The reason for this is that the \emph{array datatype is implemented as an object}.
    
    
\end{document}



