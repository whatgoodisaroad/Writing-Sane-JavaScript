\documentclass[11pt,letter]{book}

\begin{document}
    \title{Writing Sane JavaScript}
    \author{Wyatt Allen}
    \maketitle
    
    \part{Introduction}
    TODO: Title, History, Significance, Beauty, State, How to read
    
    \part{The Language}
    \chapter{The Type System}
    \section{Overview}
    JavaScript has a unique type system. In many ways it is Spartan. In many other ways 
    it is rich and expressive. In many, many other ways it is inconsistent and treacherous.
    In this chapter, you will be introduced to the type system from each of these three 
    angles.
    
    If you, the reader gain a complete understanding of these types, it will result in 
    true enjoyment of a unique and powerful system, or well-informed hatred of a tangled 
    mess. Let's find out.
    
    \section{At a Fundamental Level}
    At a fundamental level JavaScript has four basic datatypes. There are three primitive 
    types which should be familiar to you as a programmer: \texttt{boolean}, \texttt{number} 
    and \texttt{string}. The fourth datatype is \texttt{object}, which may be thought of as a 
    composite type. Although the three primitives will behave largely in a similar 
    fashion to what you're used to in other languages, some of \texttt{object}'s nuances 
    may be surprising.
    
    Although I said there were only three, JavaScript actually defines two additional primitive 
    types: \texttt{null} and \texttt{undefined}. \texttt{null} will probably look familiar to 
    you from most any object-oriented language (or C, if you want to be really technical). 
    \texttt{undefined} might be new to you. We'll explore these types in greater detail (and 
    see why they're very different) once we've dug deeper into the type system.
    
    \subsection{The \texttt{boolean} Datatype}
    If we restrict ourselves to the fundamentals of the type system, the \texttt{boolean} 
    datatype is as simple as can be. The ECMAScript standard explains it in a single sentence:
    \quote{The Boolean type represents a logical entity having two values, called \textbf{true}
    and \textbf{false}.}
    
    In practice, you'll find that \texttt{true} and \texttt{false} are a bit like keywords which 
    statically refer to the two possible boolean values (just like in most programming languages).
    \marginpar{In ECMAScript, \texttt{true} and \texttt{false} are considered ``Boolean 
    Literals.''}
    
    \subsection{The \texttt{number} Datatype}
    
    
    
\end{document}